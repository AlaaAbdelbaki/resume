%%%%%%%%%%%%%%%%%%%%%%%%%%%%%%%%%%%%%%%%%
% Developer CV
% LaTeX Class
% Version 2.0 (12/10/23)
%
% This class originates from:
% http://www.LaTeXTemplates.com
%
% Authors:
% Omar Roldan
% Based on a template by  Jan Vorisek (jan@vorisek.me)
% Based on a template by Jan Küster (info@jankuester.com)
% Modified for LaTeX Templates by Vel (vel@LaTeXTemplates.com)
%
% License:
% The MIT License (see included LICENSE file)
%
%%%%%%%%%%%%%%%%%%%%%%%%%%%%%%%%%%%%%%%%%

%----------------------------------------------------------------------------------------
%	PACKAGES AND OTHER DOCUMENT CONFIGURATIONS
%----------------------------------------------------------------------------------------
\documentclass[9pt]{developercv} % Default font size, values from 8-12pt are recommended
\usepackage{multicol}
\setlength{\columnsep}{0mm}
%----------------------------------------------------------------------------------------
\usepackage{lipsum}  
\usepackage{float}
\usepackage{tabularx}


\begin{document}


\newcommand{\printlist}[1]{%
    \foreach \x [count=\xi] in {#1} {\texttt{\x} \slashsep }%
}
%----------------------------------------------------------------------------------------
%	TITLE AND CONTACT INFORMATION
%----------------------------------------------------------------------------------------

\begin{minipage}[t]{0.5\textwidth}
    \vspace{-\baselineskip} % Required for vertically aligning minipages

    { \fontsize{16}{20} \textcolor{black}{\textbf{\MakeUppercase{Alaa Abdelbaki}}}} % First name

    \vspace{6pt}

    {\Large Étudiant en Master Cloud} % Career or current job title
\end{minipage}
\hfill
\begin{minipage}[t]{0.2\textwidth} % 20% of the page width for the first row of icons
    \vspace{-\baselineskip} % Required for vertically aligning minipages

    % The first parameter is the FontAwesome icon name, the second is the box size and the third is the text
    % \icon{Globe}{11}{\href{http://alaaabdelbaki.github.io}{alaaabdelbaki.github.io}}\\
    \icon{Phone}{11}{06 68 52 56 99}\\
    \icon{MapMarker}{11}{Avignon, France}\\

\end{minipage}
\begin{minipage}[t]{0.27\textwidth} % 27% of the page width for the second row of icons
    \vspace{-\baselineskip} % Required for vertically aligning minipages

    \icon{Envelope}{11}{\href{mailto:alaa.abdelbaki@outlook.com}{alaa.abdelbaki@outlook.com}}\\
    \icon{Github}{11}{\href{https://github.com/AlaaAbdelbaki}{github.com/AlaaAbdelbaki}}\\
    \icon{LinkedinSquare}{11}{\href{https://www.linkedin.com/in/AlaaAbdelbaki}{/in/AlaaAbdelbaki}}\\

\end{minipage}


%----------------------------------------------------------------------------------------
%	INTRODUCTION, SKILLS AND TECHNOLOGIES
%----------------------------------------------------------------------------------------

\begin{minipage}[t]{\textwidth}
    \cvsect{À propos}
    \vspace{-6pt}

    Étudiant en Master Cloud, je bénéficie de deux ans d'expérience dans le développement d'applications mobiles, principalement avec Flutter, couvrant l'ensemble du cycle, de la conception au déploiement sur Google Play et l'Apple App Store.

    À la recherche d'une alternance pour l'année universitaire 2025/2026, je souhaite intégrer un environnement dynamique dans les domaines du Cloud, DevOps ou du développement web/mobile, avec un rythme alterné de 2 semaines en entreprise et 2 semaines en formation. Motivé et curieux, je suis désireux de mettre mes compétences techniques au service d'une équipe innovante tout en continuant à développer mon expertise.
\end{minipage}
\begin{minipage}[t]{\textwidth}
    \cvsect{Compétences Techniques}\\
    \vspace{6pt}
    \textbf{Langages \& Frameworks: } \printlist{Flutter, NestJS, ExpressJS, NextJS, Shell Scripting, Java}\\
    \vspace{6pt}
    \textbf{Devops \& Cloud: } \printlist{Docker, Kubernetes, Proxmox, Jenkins}\\
    \vspace{6pt}
    \textbf{Outils \& Services: } \printlist{Firebase, GetStream, Git, Google Cloud API, Stripe}\\
    % \vspace{6pt}
    % \textbf{Systèmes d'exploitation: } \printlist{Linux, Windows, MacOS, Android, iOS}\\
    \vspace{6pt}
    \textbf{Design \& Prototypage: } \printlist{Figma, Adobe XD}
    \vspace{-12pt}
\end{minipage}
% \hfill % Whitespace between
% \begin{minipage}[t]{0.465\textwidth}
%     \cvsect{Compétences}
%     \vspace{-6pt}

%     \begin{minipage}[t]{0.2\textwidth}
%         \textbf{Languages:}
%     \end{minipage}
%     \hfill
%     \begin{minipage}[t]{0.73\textwidth}
%         Dart, Java, Kotlin, Swift, Javascript, C\#.
%     \end{minipage}
%     \vspace{4mm}

%     \begin{minipage}[t]{0.2\textwidth}
%         \textbf{Technologies:}
%     \end{minipage}
%     \hfill
%     \begin{minipage}[t]{0.73\textwidth}
%         Flutter, Git, Android, iOS, Jenkins, NestJS, ExrpessJS, NextJS, Kubernetes, Docker, Proxmox, Shell Scripting.
%     \end{minipage}

% \end{minipage}



%----------------------------------------------------------------------------------------
%	EXPERIENCE
%----------------------------------------------------------------------------------------

\vspace{10 pt}
\cvsect{Expériences professionnelles}
\begin{entrylist}
    \entry
    {07/2023 -- 08/2024}
    {Développeur Flutter}
    {\href{https://www.e-butler.com}{EButler}}
    {\vspace{-10pt}
        % Ajouter description ici ↓ et laisser cet espace ↑
        \begin{itemize}[noitemsep,topsep=0pt,parsep=0pt,partopsep=0pt, leftmargin=-1pt]
            \item Amélioration de la messagerie instantanée avec GetStream
            \item Migration vers les dernières versions de Flutter SDK
            \item Déploiement des mises à jour sur Google Play \& App Store
            \item Conception d'un nouveau design mobile \& web à partir de maquettes Figma
            \item Développement du backend avec NestJS
        \end{itemize}
        \printlist{Flutter, NestJS, Firebase, GetStream, WhatsApp Business API}}
    \entry
    {01/2022 -- 06/2023}
    {Ingénieur logiciel}
    {\href{https://www.swisspremiumnegoce.com}{Swiss Premium Negoce}}
    {\vspace{-10pt}


        \begin{itemize}[noitemsep,topsep=0pt,parsep=0pt,partopsep=0pt, leftmargin=-1pt]
            \item Conception des interfaces utilisateurs avec Adobe XD
            \item Développement full-stack (Flutter + ExpressJS)
            \item Intégration des API Google Cloud (Maps, Directions)
            \item Mise en place de Stripe pour les paiements
            \item Déploiement sur Google Play \& App Store
        \end{itemize}
        \printlist{Flutter, ExpressJS, NextJS, Firebase, Google Cloud API, Stripe}}
\end{entrylist}

%----------------------------------------------------------------------------------------
%	EDUCATION
%----------------------------------------------------------------------------------------
\vspace{-10 pt}
\cvsect{Parcours académique}
\begin{entrylist}
    \entry
    {09/2024 - 06/2026}
    {Master Informatique - Infrastructures Cloud \& Systèmes Distribués (SYRIUS)
        \vspace{-10pt}
    }
    {\href{https://formations.univ-avignon.fr/formation/master-informatique-systemes-informatiques-communicants-reseaux-services-et-securite-sicom/}{Université d'Avignon}}
    {
        % Formation axée sur le cloud, les systèmes distribués et les pratiques DevOps, avec un projet en intelligence artificielle portant sur la génération de mots de passe à partir de données compromises, soulignant les risques liés au réemploi d'identifiants.
        \begin{itemize}[noitemsep,topsep=0pt,parsep=0pt,partopsep=0pt, leftmargin=0pt]
            \item Virtualisation, conteneurisation et orchestration.
            \item Intelligence artificielle.
            \item Sécurité des systèmes et cryptographie.
        \end{itemize}
    }

    \entry
    {09/2018 - 06/2022}
    {Diplome National d'ingénieur en informatique}
    {\href{https://www.esprit.tn}{Ecole Sup. Privée d'Informatique et de Technologies (ESPRIT)}
        \vspace{-10pt}
    }
    {
        % Formation d'ingénieur axée sur les fondamentaux de l'informatique, couvrant le développement logiciel, l'intelligence artificielle et les pratiques DevOps, avec la réalisation de projets concrets dans les domaines mobile et web.
        \begin{itemize}[noitemsep,topsep=0pt,parsep=0pt,partopsep=0pt, leftmargin=0pt]
            \item Développement d'applications Android et iOS.
            \item Développement web.
            \item Intelligence artificielle.
            \item Vision par ordinateur.
            \item DevOps.
        \end{itemize}
    }
\end{entrylist}

%----------------------------------------------------------------------------------------
%	Projects
%----------------------------------------------------------------------------------------
\vspace{-20 pt}
\cvsect{Projets}
\begin{entrylist}
    \entry
    {2023}
    {Elisys}
    {
        \href{https://play.google.com/store/apps/details?id=fr.elisys.mobile&hl=en}{Google Play}
    }
    {
        Application de gestion de dépôt permettant aux techniciens de gérer leurs produits en scanannt leurs QR codes.
    }
    \entry
    {2024}
    {\href{https://mazadlive.net/en}{Mazad Live}}
    {
        \href{https://play.google.com/store/apps/details?id=mazad.live.app&hl=en}{Google Play} \& \href{https://apps.apple.com/tn/app/mzad-live/id6463001872}{App Store}
    }
    {
        Une platforme de vente aux enchères en ligne, développée avec Flutter, permettant aux utilisateurs de participer à des enchères en direct sur des produits variés.
    }
\end{entrylist}




%----------------------------------------------------------------------------------------
%	LANGUAGES
%----------------------------------------------------------------------------------------
\vspace{-20 pt}
\cvsect{Langues}
\vspace{-6pt}


\begin{itemize}[noitemsep,topsep=0pt,parsep=0pt,partopsep=0pt, leftmargin=0.18\textwidth]
    \item \textbf{Français} - Niveau avancé
    \item \textbf{Anglais} - Niveau avancé
    \item \textbf{Arabe} - Langue maternelle
\end{itemize}



% \hspace{26mm} \textbf{Français} - Niveau avancé, \textbf{ Anglais} - Niveau avancé, \textbf{ Arabe} - Langue maternelle

%----------------------------------------------------------------------------------------

\end{document}